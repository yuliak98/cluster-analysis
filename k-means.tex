\documentclass[12pt, a4paper]{article}
\usepackage{tikz}
\usepackage{cmap} 
\usepackage[T2A]{fontenc} 
\usepackage[utf8]{inputenc} 
\usepackage[english,russian]{babel}
\usepackage{amsmath}

\title{Алгоритмы кластеризации}
\date{}
\begin{document}
\maketitle

\section*{K-means}
Алгоритм разбивает множество элементов на известное число кластеров. Основной идеей алгоритма является минимизировать среднее расстояние между каждым из объектов в кластере и его центроидом. 
\\Определим центроид \( \mu(\omega) \) кластера \(\omega \in \varOmega \) как: 
\[
        \mu(\omega) = \frac{1}{|\omega|} \sum_{s \in \omega} s
\]
где \( \varOmega \) = \{ \( \omega_1, \omega_2, \ldots, \omega_k \) \} -- множество кластеров и \( S = \{ s_1, s_2, \ldots , s_n \} \) -- множество объектов. 
\begin{enumerate}
\item На первом шаге фиксируются начальное приближение для центроидов кластеров, а объекты разбиваются на кластеры при условии минимизации внутрекластерного расстояния. Т.е. на этом шаге создаем кластеры, при условии существования центроида. Внутрикластерное расстояние можно определить по формуле:
\[
             d = \frac{1}{N} \sum_{k=1}^K \sum_{s \in \omega_k} \|s - \mu(\omega_k)\|^2
\]

\item Далее фиксируется распределение по кластерам и пересчитываются значения центроидов. 
Алгоритм продолжается до тех пор, пока не будет выполняться какой-то критерий остановки.
\end{enumerate}

Примеры критериев остановки: 


\begin{itemize} 
\item Кластеры не изменились после предыдущей итерации.

 
\item Фиксированное число итераций.


\item Центроиды не изменились после предыдущей итерации.
\end{itemize}

Начальное приближение можно выбрать разными способами:


\begin{itemize} 
\item Использовать результат другого алгоритма кластеризации.


\item Исключение выбросов из множества при выборе начального положения. 


\item Запустить алгоритм из различных начальных приближений и выбрать наилучший.
\end{itemize}

\section*{Пошаговая иллюстрация}
\begin{tikzpicture}
    \begin{scope}
        \draw (-3.5, 3.5 ) -- (3.5, 3.5);
        \draw (-3.5, -3.5) -- (3.5, -3.5);
        \draw (-3.5, -3.5) -- (-3.5, 3.5);
        \draw (3.5, 3.5) -- (3.5, -3.5);

        \draw[red, fill=red] (-1.5, -2.5) circle (0.2);
        \draw[red, fill=red] (-1.75, -0.5) circle (0.2);
        \draw[red, fill=red] (-1.5, 1.4) circle (0.2);
        \draw[red, fill=red] (-2, 0.75) circle (0.2);
        \draw[red, fill=red] (-2.625, 1.8) circle (0.2);
        \draw[yellow, fill=yellow] (1.75, -3.2) circle (0.2);
        \draw[yellow, fill=yellow] (3, -2.8) circle (0.2);
        \draw[yellow, fill=yellow] (3, -0.5) circle (0.2);
        \draw[yellow, fill=yellow] (1.75, -0.5) circle (0.2);
        \draw[yellow, fill=yellow] (2.7, 1.2) circle (0.2);
        \draw[yellow, fill=yellow] (2.5, 1.8) circle (0.2);
        \draw[yellow, fill=yellow] (0.5, 1) circle (0.2);
        \draw[yellow, fill=yellow] (-0.4, 2.3) circle (0.2);
        \draw[yellow, fill=yellow] (-0.9, 2.9) circle (0.2); 
        \draw[red, fill=red] (-3, 1) -- (-2.75, 1.5) -- (-2.5, 1) -- cycle;
        \draw[blue, fill=red] (-2.5, -0.5) -- (-2.25, 0) -- (-2, -0.5) -- cycle;
        \draw[yellow, fill=yellow] (0.25, 3) -- (.5, 3.5) -- (0.75, 3) -- cycle;
        \draw[blue, fill=yellow] (1, 0.2) -- (1.25, 0.7) -- (1.5, 0.2) -- cycle;
        \draw[blue] (1.5, -3.4) -- (-1.5, 3.4);
        \draw[->][blue] (-2.75, 0.9) -- (-2.5, -0.1);
        \draw[->][blue] (0.5, 2.9) -- (1, 0.6);
        
    \end{scope}
    \begin{scope}[xshift=7.5cm]
        \draw (-3.5, 3.5 ) -- (3.5, 3.5);
        \draw (-3.5, -3.5) -- (3.5, -3.5);
        \draw (-3.5, -3.5) -- (-3.5, 3.5);
        \draw (3.5, 3.5) -- (3.5, -3.5);
    
        \draw[red, fill=red] (-1.5, -2.5) circle (0.2);
        \draw[red, fill=red] (-1.75, -0.5) circle (0.2);
        \draw[red, fill=red] (-1.5, 1.4) circle (0.2);
        \draw[red, fill=red] (-2, 0.75) circle (0.2);
        \draw[red, fill=red] (-2.625, 1.8) circle (0.2);
        \draw[yellow, fill=yellow] (1.75, -3.2) circle (0.2);
        \draw[yellow, fill=yellow] (3, -2.8) circle (0.2);
        \draw[yellow, fill=yellow] (3, -0.5) circle (0.2);
        \draw[yellow, fill=yellow] (1.75, -0.5) circle (0.2);
        \draw[yellow, fill=yellow] (2.7, 1.2) circle (0.2);
        \draw[yellow, fill=yellow] (2.5, 1.8) circle (0.2);
        \draw[yellow, fill=yellow] (0.5, 1) circle (0.2);
        \draw[red, fill=red] (-0.4, 2.3) circle (0.2);
        \draw[red, fill=red] (-0.9, 2.9) circle (0.2); 
        \draw[blue, fill=red] (-2, 1) -- (-1.75, 1.5) -- (-1.5, 1) -- cycle;
        \draw[red, fill=red] (-2.5, -0.5) -- (-2.25, 0) -- (-2, -0.5) -- cycle;
        \draw[blue, fill=yellow] (1.75, -1) -- (2, -0.5) -- (2.25, -1) -- cycle;
        \draw[yellow, fill=yellow] (1, 0.2) -- (1.25, 0.7) -- (1.5, 0.2) -- cycle;
        \draw[blue] (0.5, 3.4) -- (-1.5, -3.4);
        \draw[->][blue] (-2.1, -0.1) -- (-1.7, 0.9);
        \draw[->][blue] (1.25, 0.1) -- (1.8, -0.6);
    \end{scope}
    \begin{scope}[yshift=-7.5cm]
        \draw (-3.5, 3.5 ) -- (3.5, 3.5);
        \draw (-3.5, -3.5) -- (3.5, -3.5);
        \draw (-3.5, -3.5) -- (-3.5, 3.5);
        \draw (3.5, 3.5) -- (3.5, -3.5);
    
        \draw[yellow, fill=yellow] (-1.5, -2.5) circle (0.2);
        \draw[red, fill=red] (-1.75, -0.5) circle (0.2);
        \draw[red, fill=red] (-1.5, 1.4) circle (0.2);
        \draw[red, fill=red] (-2, 0.75) circle (0.2);
        \draw[red, fill=red] (-2.625, 1.8) circle (0.2);
        \draw[yellow, fill=yellow] (1.75, -3.2) circle (0.2);
        \draw[yellow, fill=yellow] (3, -2.8) circle (0.2);
        \draw[yellow, fill=yellow] (3, -0.5) circle (0.2);
        \draw[yellow, fill=yellow] (1.75, -0.5) circle (0.2);
        \draw[yellow, fill=yellow] (2.7, 1.2) circle (0.2);
        \draw[yellow, fill=yellow] (2.5, 1.8) circle (0.2);
        \draw[red, fill=red] (0.5, 1) circle (0.2);
        \draw[red, fill=red] (-0.4, 2.3) circle (0.2);
        \draw[red, fill=red] (-0.9, 2.9) circle (0.2); 
        \draw[red, fill=red] (-1, 1.5) -- (-0.75, 2) -- (-0.5, 1.5) -- cycle;
        \draw[yellow, fill=yellow] (1.5, -1.5) -- (1.75, -1) -- (2, -1.5) -- cycle;
        \draw[blue] (3, 3.4) -- (-2.8, -3.4);
        

    \end{scope}
    \begin{scope}[yshift=-7.5cm, xshift=7.5cm]
        \draw (-3.5, 3.5 ) -- (3.5, 3.5);
        \draw (-3.5, -3.5) -- (3.5, -3.5);
        \draw (-3.5, -3.5) -- (-3.5, 3.5);
        \draw (3.5, 3.5) -- (3.5, -3.5);
    
        \draw[yellow, fill=yellow] (-1.5, -2.5) circle (0.2);
        \draw[red, fill=red] (-1.75, -0.5) circle (0.2);
        \draw[red, fill=red] (-1.5, 1.4) circle (0.2);
        \draw[red, fill=red] (-2, 0.75) circle (0.2);
        \draw[red, fill=red] (-2.625, 1.8) circle (0.2);
        \draw[yellow, fill=yellow] (1.75, -3.2) circle (0.2);
        \draw[yellow, fill=yellow] (3, -2.8) circle (0.2);
        \draw[yellow, fill=yellow] (3, -0.5) circle (0.2);
        \draw[yellow, fill=yellow] (1.75, -0.5) circle (0.2);
        \draw[yellow, fill=yellow] (2.7, 1.2) circle (0.2);
        \draw[yellow, fill=yellow] (2.5, 1.8) circle (0.2);
        \draw[red, fill=red] (0.5, 1) circle (0.2);
        \draw[red, fill=red] (-0.4, 2.3) circle (0.2);
        \draw[red, fill=red] (-0.9, 2.9) circle (0.2); 
        \draw[red, fill=red] (-2, 1) -- (-1.75, 1.5) -- (-1.5, 1) -- cycle;
        \draw[blue, fill=red] (-1, 1.5) -- (-0.75, 2) -- (-0.5, 1.5) -- cycle;
        \draw[yellow, fill=yellow] (1.75, -1) -- (2, -0.5) -- (2.25, -1) -- cycle;
        \draw[blue, fill=yellow] (1.5, -1.5) -- (1.75, -1) -- (2, -1.5) -- cycle;
        \draw[blue] (3, 3.4) -- (-2.5, -3.4);
        \draw[->][blue] (-1.4, 1.15) -- (-0.8, 1.4);
        \draw[->][blue] (2, -1.05) -- (1.9, -1.25);
        
    \end{scope}

\end{tikzpicture}

\end{document}